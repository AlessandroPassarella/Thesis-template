\chapter{Conclusioni}
\label{cap:conclusioni}

\intro{Nel capitolo finale è presente un resoconto delle attività svolte e degli obiettivi raggiunti. Inoltre è presente una valutazione personale dello stage e delle conoscenze acquisite}

\section{Consuntivo finale}

Nella Tabella 7.1 è presente la valutazione oraria finale.

\begin{table}[H]
    \centering
    \begin{tabular}{|l|c|c|}
    \hline
    \textbf{Attività} & \textbf{Ore previste} & \textbf{Ore svolte} \\  \hline
    Studio della letteratura & 30 & 40 \\  \hline
    Studio delle tecnologie & 10 & 10 \\  \hline
    Definizione e analisi del problema & 60 & 44 \\  \hline
    Progetto e sviluppo algoritmo genetico & 130 & 120 \\  \hline
    Test e analisi statistica & 44 & 70 \\  \hline
    Stesura documentazione codice & 30 & 20 \\  \hline
    \textbf{Totale} & \textbf{304} & \textbf{304} \\  \hline
    \end{tabular}
    \caption{Consuntivo finale delle attività}
\end{table}

Il livello dell'argomento trattato nel campo della Ricerca Operativa ha richiesto l'impiego di più tempo per prendere familiarità con le nozioni di base necessarie, le fasi di sviluppo e analisi degli algoritmi esistenti sono state più agevoli. Infine, la fase di test ha richiesto molto più tempo del previsto a causa dei tempi necessari per la calibrazione dei parametri del GA della configurazione. La stesura della documentazione ha invece richiesto meno tempo del previsto. Complessivamente, i tempi totali sono stati rispettati.

\section{Raggiungimento degli obiettivi}

Nella Tabella 7.2 sono elencati gli obiettivi e il raggiungimento o meno di ciascuno di essi.

\begin{table}[H]
    \centering
    \begin{tabular}{|c|c|}
    \hline
    \textbf{Obiettivo} & \textbf{Raggiunto} \\ \hline
    OB1 & Si \\ \hline
    OB2 & Si \\ \hline
    OB3 & Si \\ \hline
    OB4 & Si \\ \hline
    OB5 & Si \\ \hline
    OB6 & Si \\ \hline
    OB6 & Si \\ \hline
    DE1 & Si \\ \hline
    FA1 & No \\ \hline
    \end{tabular}
    \caption{Raggiungimento obiettivi}
\end{table}

Sono stati raggiunti tutti gli obiettivi obbligatori e desiderabili. Il non raggiungimento dell'obbiettivo facoltativo è dovuto a mancanza di ulteriore tempo.

\section{Conoscenze acquisite}

Lavorando in Salvagnini, ho potuto approfondire le dinamiche tipiche di un contesto progettuale e comprendere le sfide legate all'ottimizzazione e alla gestione delle risorse in ambito aziendale. Ho acquisito una chiara consapevolezza di quanto siano cruciali fattori come il tempo, la sincronizzazione e l'efficienza per il successo operativo. In un ambiente competitivo, il miglioramento continuo è una priorità fondamentale, e ciò richiede la capacità di analizzare con attenzione le esigenze e le criticità, adottando un approccio creativo e proattivo per proporre soluzioni innovative e sostenibili.

Dal punto di vista informatico, questa esperienza mi ha permesso di approfondire le mie competenze in Python, sfruttando la sua potenza per sviluppare e ottimizzare algoritmi genetici. Ho potuto applicare in modo concreto le nozioni teoriche apprese durante il mio percorso di studi, specialmente nell'ambito della Ricerca Operativa, traducendole in soluzioni pratiche per problemi reali. L'implementazione dell'algoritmo genetico mi ha permesso di esplorare concetti avanzati di ottimizzazione e di toccare con mano l’efficacia di questi strumenti nel risolvere problemi complessi.

Questa esperienza mi ha dato anche l'opportunità di migliorare le mie capacità di programmazione e di approfondire l'importanza della creatività e dell'innovazione nel mondo dello sviluppo software, fornendomi una base solida per affrontare progetti futuri con un approccio analitico e orientato al risultato.

\section{Valutazione personale}

La complessità del progetto e la necessità di affacciarmi a un mondo, quello degli algoritmi evolutivi, che non conoscevo mi ha inizialmente richiesto del tempo per abituarmi. Tuttavia, questa sfida si è rivelata un'opportunità unica per approfondire la mia conoscenza di Python e delle tecniche di ottimizzazione, scoprendo tecnologie e approcci che non conoscevo. Il bisogno di approfondire la conoscenza teorica e di scegliere i corretti elementi per lo sviluppo dell'algoritmo genetico ha sicuramente potenziato le mie capacità di problem solving e il mio spirito critico.

L'esperienza maturata in azienda ha un riscontro assolutamente positivo, da subito mi sono trovato in un ambiente di lavoro sereno e stimolante. La collaborazione con il tutor aziendale è stata importantissima, mi ha permesso di affacciarmi a un mondo lavorativo professionale e una visone del progetto molto più analitica.

Riguardo al progetto, ritengo che l'algoritmo genetico abbia un potenziale notevole per risolvere problemi complessi, ma che necessiti di un'attenta analisi per essere applicato al meglio in contesti reali. È un approccio che, pur essendo molto potente, richiede una comprensione profonda dei problemi da risolvere e delle soluzioni da implementare. Per ottenere i migliori risultati, sarebbe utile il coinvolgimento di un team con competenze diversificate e di esperti che possano guidare la scelta delle strategie più efficaci in base agli obiettivi specifici.

Nel complesso, questa esperienza ha avuto un esito estremamente positivo. Mi ha permesso di raggiungere risultati concreti, di ampliare le mie competenze in ambito di ottimizzazione e programmazione, e di comprendere meglio le necessità di un progetto complesso.