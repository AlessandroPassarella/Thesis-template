\chapter{Introduzione}
\label{cap:introduzione}

\intro{In questo capitolo verrà descritto brevemente il contesto in cui si è svolto lo stage, e come questo si è inserito nell’ottica aziendale.} \\

% \noindent Esempio di utilizzo di un termine nel glossario \\
% \gls{api}. \\

% \noindent Esempio di citazione in linea \\
% \cite{site:agile-manifesto}. \\

% \noindent Esempio di citazione nel pie' di pagina \\
% citazione\footcite{womak:lean-thinking} \\

\section{L'azienda}

Il lavoro svolto in questa tesi fa riferimento al periodo di stage svolto presso Salvagnini Italia S.p.A. (da ora in poi abbreviata in Salvagnini). L'azienda, situata nel nord Italia, progetta, produce e commercializza Sistemi di Produzione Flessibili (FMS) costituiti da macchine a Controllo Numerico Computerizzato (CNC) per il taglio e la piegatura della lamiera, insieme a software dedicati. I processi in un FMS Salvagnini sono progettati per garantire una produzione personalizzata, massimizzando la produttività e minimizzando sprechi di materiali e consumi. In questo contesto, sorgono numerosi problemi di ottimizzazione, gestiti dal dipartimento di Ricerca e Sviluppo (R\&S).

Da un lato, le macchine sono dotate di un numero crescente di processi automatizzati che eliminano i tempi morti e riducono le possibilità di errore. Dall'altro, una gran parte dell'ottimizzazione viene affrontata a livello software, attraverso lo sviluppo di applicazioni CAD (Computer-Aided Design) e CAM (Computer-Aided Manufacturing). Queste applicazioni assistono gli operatori con strumenti automatici per affrontare problemi complessi, come il modo in cui una macchina deve lavorare le lamiere, l'ordine in cui devono essere eseguite le operazioni, e così via.

\section{Scopo dello stage}

L’attività di stage prevede lo studio, sviluppo e test di un algoritmo di ottimizzazione per la risoluzione di problemi di nesting di interesse nell’ambiente Salvagnini.
Occorre affrontare inizialmente lo studio della letteratura di Ricerca Operativa nell’ambito dell’algoritmo genetico, un metodo evolutivo che classicamente ha riportato risultati significativi in problemi affini ai suddetti. Si passa poi alla definizione del problema di nesting in Salvagnini e all’analisi del codice sviluppato internamente per la risoluzione dello stesso. Avviene poi lo sviluppo di una versione dell’algoritmo genetico che risolva il problema di nesting in esame e che allo stesso tempo si integri nell’attuale software di ottimizzazione. Infine, verrà eseguita un’analisi statistica dei risultati prodotti da tale algoritmo, confrontandolo con altri metodi risolutivi già sviluppati dall’azienda, in termini di tempo di esecuzione e di qualità delle soluzioni proposte.

% \section{Organizzazione del testo}

% \begin{description}
%     \item[{\hyperref[cap:processi-metodologie]{Il secondo capitolo}}] descrive ...
    
%     \item[{\hyperref[cap:descrizione-stage]{Il terzo capitolo}}] approfondisce ...
    
%     \item[{\hyperref[cap:analisi-requisiti]{Il quarto capitolo}}] approfondisce ...
    
%     \item[{\hyperref[cap:progettazione-codifica]{Il quinto capitolo}}] approfondisce ...
    
%     \item[{\hyperref[cap:verifica-validazione]{Il sesto capitolo}}] approfondisce ...
    
%     \item[{\hyperref[cap:conclusioni]{Nel settimo capitolo}}] descrive ...
% \end{description}

% Riguardo la stesura del testo, relativamente al documento sono state adottate le seguenti convenzioni tipografiche:
% \begin{itemize}
% 	\item gli acronimi, le abbreviazioni e i termini ambigui o di uso non comune menzionati vengono definiti nel glossario, situato alla fine del presente documento;
% 	\item per la prima occorrenza dei termini riportati nel glossario viene utilizzata la seguente nomenclatura: \emph{parola}\glsfirstoccur;
% 	\item i termini in lingua straniera o facenti parti del gergo tecnico sono evidenziati con il carattere \emph{corsivo}.
% \end{itemize}
