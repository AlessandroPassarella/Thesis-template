\chapter{Introduzione}
\label{cap:introduzione}

\intro{In questo capitolo verrà descritto brevemente il contesto in cui si è svolto lo stage, e come questo si è inserito nell’ottica aziendale.} \\

\noindent Esempio di utilizzo di un termine nel glossario \\
\gls{api}. \\

\noindent Esempio di citazione in linea \\
\cite{site:agile-manifesto}. \\

\noindent Esempio di citazione nel pie' di pagina \\
citazione\footcite{womak:lean-thinking} \\

\section{L'azienda}

Da oltre sessant’anni Salvagnini progetta, produce e vende macchine utensili, sistemi industriali e automazioni flessibili per la lavorazione della lamiera. Pannellatrici, punzonatrici, laser in fibra, presse piegatrici, sistemi FMS, magazzini automatici e software a marchio Salvagnini sono utilizzati in moltissimi settori applicativi diversi.

Ad oggi conta:
25 filiali,
5 stabilimenti produttivi,
oltre 2000 dipendenti,
500 milioni di fatturato e più di 8.000 installazioni in 84 paesi.\\
Salvagnini ha deciso di sviluppare i suoi software internamente in modo da non doversi appoggiare a software house, questo garantisce una personalizzazione ad hoc, secondo le esigenze di ogni cliente, più rapida ed efficace data la già approfondita conoscenza del software aziendale, inoltre, questo crea un ambiente di lavoro altamente specializzato.

\section{Scopo dello stage}

L’attività di stage prevede lo studio, sviluppo e test di un algoritmo di ottimizzazione per la risoluzione di problemi di nesting di interesse nell’ambiente Salvagnini.
Occorre affrontare inizialmente lo studio della letteratura di Ricerca Operativa nell’ambito dell’algoritmo genetico, un metodo evolutivo che classicamente ha riportato risultati significativi in problemi affini ai suddetti. Si passa poi alla definizione del problema di nesting in Salvagnini e all’analisi del codice sviluppato internamente per la risoluzione dello stesso. Avviene poi lo sviluppo di una versione dell’algoritmo genetico che risolva il problema di nesting in esame e che allo stesso tempo si integri nell’attuale software di ottimizzazione. Infine, verrà eseguita un’analisi statistica dei risultati prodotti da tale algoritmo, confrontandolo con altri metodi risolutivi già sviluppati dall’azienda, in termini di tempo di esecuzione e di qualità delle soluzioni proposte.

\section{Organizzazione del testo}

\begin{description}
    \item[{\hyperref[cap:processi-metodologie]{Il secondo capitolo}}] descrive ...
    
    \item[{\hyperref[cap:descrizione-stage]{Il terzo capitolo}}] approfondisce ...
    
    \item[{\hyperref[cap:analisi-requisiti]{Il quarto capitolo}}] approfondisce ...
    
    \item[{\hyperref[cap:progettazione-codifica]{Il quinto capitolo}}] approfondisce ...
    
    \item[{\hyperref[cap:verifica-validazione]{Il sesto capitolo}}] approfondisce ...
    
    \item[{\hyperref[cap:conclusioni]{Nel settimo capitolo}}] descrive ...
\end{description}

Riguardo la stesura del testo, relativamente al documento sono state adottate le seguenti convenzioni tipografiche:
\begin{itemize}
	\item gli acronimi, le abbreviazioni e i termini ambigui o di uso non comune menzionati vengono definiti nel glossario, situato alla fine del presente documento;
	\item per la prima occorrenza dei termini riportati nel glossario viene utilizzata la seguente nomenclatura: \emph{parola}\glsfirstoccur;
	\item i termini in lingua straniera o facenti parti del gergo tecnico sono evidenziati con il carattere \emph{corsivo}.
\end{itemize}
