\chapter{Pianificazione dello stage}
\label{cap:processi-metodologie}

\intro{In questo capitolo verranno riportati in modo approfondito gli obiettivi dello stage, contestualizzazione delle attività alla realtà aziendale e scadenzario delle stesse.}\\


\section{Obiettivi}

Ogni obiettivo è provvisto di un codice identificativo formato da una delle seguenti stringhe \textbf{OB}, \textbf{DE} e \textbf{FA}, che rappresentano il livello di importanza e da un numero incrementale positivo, che rispetta la seguente nomenclatura:
\begin{center}
    [Importanza][Identificativo]
\end{center}
Gli obiettivi rappresentano ciò che vogliamo ottenere dell'attività di stage e sono divisi, a seconda dell'importanza, come segue:
\begin{itemize}
    \item Obiettivi obbligatori \textbf{(OB)}: 
    Rappresentano gli obiettivi vincolanti che dovranno essere necessariamente soddisfatti;
    \item Obiettivi desiderabili \textbf{(DE)}:
    Rappresentano gli obiettivi non vincolanti ma dal riconoscibile valore aggiunto;
    \item Obiettivi facoltativi \textbf{(FA)}:
    Rappresentano gli obiettivi facoltativi non vincolanti ma dal valore aggiunto non strettamente competitivo.
\end{itemize}
Le seguenti Tabelle 2.1, 2.2 e 2.3 rappresentano rispettivamente, gli obiettivi obbligatori, opzionali e facoltativi.
Ogni obiettivo è affiancato da una breve descrizione.


\begin{table}[H]
\centering
\begin{tabular}{|l|p{10cm}|}
\hline
\textbf{Obiettivo} & \textbf{Descrizione} \\ \hline
OB1 & Studio e conoscenza di algoritmi euristici e metaeuristici nella letteratura della Ricerca Operativa \\ \hline
OB2 & Definizione del problema di nesting in esame \\ \hline
OB3 & Conoscenza del \emph{framework}\glsfirstoccur fornito dall’azienda con funzionalità di natura geometrica e di ottimizzazione nel contesto del nesting \\ \hline
OB4 & Implementazione di un algoritmo genetico per la risoluzione del problema di nesting in esame \\ \hline
OB5 & Verifica su \emph{benchmark}\glsfirstoccur aziendali dell’algoritmo, confrontandolo con soluzioni ottenute da altri metodi risolutivi interni all’azienda \\ \hline
OB6 & Documentazione del codice e manuale di utilizzo dei moduli sviluppati \\ \hline
\end{tabular}
\caption{Obiettivi Obbligatori}
\end{table}

\begin{table}[H]
\centering
\begin{tabular}{|l|p{10cm}|}
\hline
\textbf{Obiettivo} & \textbf{Descrizione} \\ \hline
DE1 & Analisi statistica per una validazione rigorosa e valutazione della significatività dei risultati ottenuti \\ \hline
\end{tabular}
\caption{Obiettivi Desiderabili}
\end{table}

\begin{table}[H]
\centering
\begin{tabular}{|l|p{10cm}|}
\hline
\textbf{Obiettivo} & \textbf{Descrizione} \\ \hline
FA1 & Adattamento e test dell'algoritmo genetico sviluppato su problemi di nesting simili a quello definito (ad esempio problemi con forme regolari/irregolari, strip packing) \\ \hline
\end{tabular}
\caption{Obiettivi Facoltativi}
\end{table}

\section{Vincoli temporali e tecnologici}

L'unico vincolo tecnologico imposto è stato lo sviluppo dell'algoritmo tramite il linguaggio \emph{Python}\glsfirstoccur. Questo perché Python rendere il processo di sviluppo più efficiente, infatti, è particolarmente adatto al \emph{fast prototyping}\glsfirstoccur, dato che non necessita di compilazione, a differenza di linguaggi come C++ e Java, e, inoltre, permette di avere maggiore flessibilità del codice grazie anche dall'uso dei \emph{notebook}\glsfirstoccur.

Prima di iniziare lo stage, è stato concordato con l’azienda un piano di lavoro su un totale di 304 ore, lavorando 5 giorni a settimana, 8 ore per ciascun giorno eccetto il venerdì che per politica aziendale prevede 6,5 ore. 

\section{Pianificazione}

La Tabella 2.4 indica la divisione delle attività da svolgere con una stima in termini di ore della durata di ciascuna attività.\\

\begin{table}[H]
\centering
\begin{tabular}{|l|p{10cm}|}
\hline
\textbf{Durata in ore} & \textbf{Descrizione dell'attività} \\ \hline
30 & Studio della letteratura di Ricerca Operativa nell’ambito delle euristiche di ottimizzazione combinatoria e degli algoritmi genetici \\ \hline
10 & Studio delle tecnologie necessarie allo sviluppo dell’algoritmo di ottimizzazione (python, librerie di ottimizzazione) \\ \hline
60 & Definizione del problema di nesting e analisi degli algoritmi esistenti e della loro implementazione \\ \hline
130 & Sviluppo algoritmo genetico:
\begin{itemize}
    \item Progettazione degli operatori di codifica e decodifica delle soluzioni basati su euristiche costruttive (30 ore)
    \item Progettazione degli operatori di fitness, selezione e rimpiazzo generazionale (10 ore)
    \item Implementazione delle componenti algoritmiche (80 ore)
    \item Test degli algoritmi implementati su istanze di prova (10 ore)
\end{itemize} \\ \hline
44 & Test dell'algoritmo sviluppato su benchmark di letteratura e aziendali. Analisi statistica dei risultati prodotti su diversi benchmark confrontandoli con quelli dei metodi sviluppati dall'azienda \\ \hline
30 & Stesura documentazione codice e manuale di utilizzo dei modelli sviluppati \\ \hline
\textbf{Totale: 304 ore} & \\ \hline
\end{tabular}
\caption{Pianificazione del lavoro}
\end{table}

\section{Ambiente di lavoro}

Di seguito verranno esposti brevemente il ciclo di vita e la gestione del progetto, considerando anche l'ambiente di sviluppo.
\subsection{Metodo di sviluppo}
Il ciclo di vita dell'algoritmo he seguito i seguenti passi:
\begin{itemize}
    \item Comprensione del problema: definizione del problema di nesting;
    \item Analisi letteratura: lettura articoli accademici di algoritmi genetici;
    \item Sviluppo: implementazione dell'algoritmo;
    \item Verifica: testing massivo e verifica della soluzione fornita.
\end{itemize}

\subsection{Gestione di progetto} \hypertarget{siti}{}

In generale, per la gestione della comunicazione, condivisione di informazioni, documentazione e articoli si è fatto uso dell’applicazione \emph{Microsoft Teams}\glsfirstoccur [\hyperlink{bibliografia}{7}], per il versionamento e la gestione dei task si è fatto uso del servizio \emph{GitHub}\glsfirstoccur [\hyperlink{bibliografia}{2}], infine per quanto riguarda l’interfaccia con cui versionare ho utilizzato \emph{GitHub Desktop}\glsfirstoccur [\hyperlink{bibliografia}{3}].

\subsection{Linguaggio di programmazione e ambiente di sviluppo}

Per la fase iniziale dell'implementazione si è lavorato utilizzando \emph{Jupyter Notebook}\glsfirstoccur, con il quale è stato possibile scrivere metodi in linguaggio Python in modo molto agevole. Quando l'algoritmo ha raggiunto una certa completezza, è stato necessario raggruppare i vari metodi in più classi e iniziare a lavorare su VisualStudio Code all'algoritmo e su Jupyter Notebook ai test da effettuare.
Python è un linguaggio di programmazione orientato agli oggetti ed interpretato dinamicamente al momento dell’esecuzione da un interprete. Python risulta veramente versatile in quanto fornisce incredibili funzionalità utilizzabili in modo semplice e intuitivo, inoltre, dispone di moltissime librerie che permettono le più svariate operazioni, come: metodi per l'analisi statistica, la manipolazione dei dati, la creazione di grafici e operazioni sui file. 
Le principali librerie utilizzate sono: random (per l'utilizzo delle funzioni randomiche), pandas (per la manipolazione e analisi dei dati), matplotlib (per la visualizzazione dei risultati ottenuti nei test) e statsmodel (per l'analisi statistica).
\emph{VisualStudio Code}\glsfirstoccur è stato scelto per rendere agevole l'implementazione è stato utilizzato infatti, grazie alle sue estensioni, risulta molto versatile e completo. In particolare sono state usate estensioni generiche per python ma anche per l'uso dei notebook direttamente da VSCode e per la gestione dei file formato \emph{csv}\glsfirstoccur, usati nella fase di test.

\section{Analisi dei rischi}

Questa sezione del documento si concentra sull’analisi delle potenziali difficoltà che potrebbero emergere durante il corso dello stage, con l’obiettivo di identificare, analizzare e prevenire eventuali ostacoli o rallentamenti che potrebbero comprometterne il progresso complessivo. Per affrontare queste possibili problematiche, si è scelto di esaminare attentamente ciascun rischio, fornendo dettagli quali la descrizione del rischio, il grado di rischio associato, la pericolosità, le precauzioni da adottare e una possibile soluzione. Il grado di rischio definisce la possibilità di occorrenza dello stesso e varia da una scala crescente da 1 a 5, mentre la pericolosità varia tra “Alta”, “Media” e “Bassa”. 

Le informazioni vengono presentante in forma tabellare, in modo da facilitarne il monitoraggio continuo durante l’intero ciclo di vita del progetto. Ogni rischio è provvisto di un codice identificativo formato da una delle seguenti stringhe \textbf{RP}, \textbf{RO} e \textbf{RT}, che rappresentano la categoria di rischo e da un numero incrementale positivo. Le principali categorie di rischi considerate includono:
\begin{itemize}
    \item Rischi personali \textbf{(RP)};
    \item Rischi organizzativi \textbf{(RO)};
    \item Rischi tecnologici/software \textbf{(RT)}.
\end{itemize}
I rischi rispettano la seguente nomenclatura:
\begin{center}
    [Categoria][Identificativo]
\end{center}

Le seguenti Tabelle da 2.5 a 2.11 contengono l'analisi di ciascun rischio.

\risk{RP1 - Mancanza di competenze tecniche specifiche}
{Mancanza di esperienze professionali nella realizzazione dell'algoritmo. Mancanza di conoscenze delle tecnologie necessarie allo sviluppo del prodotto.}
{2}
{Alta}
{Fase iniziale di studio delle tecnologie.}
{Comunicazione con tutor aziendale per concordare misure di adattamento.}

\risk{RP2 - Mancanza di conoscenze teoriche specifiche}
{Mancanza di conoscenze relative al problema in questione. Mancanza di conoscenza specifiche dell'ambito industriale.}
{5}
{Media}
{Fase iniziale di studio del problema di nesting e dell'approccio aziendale.}
{Comunicazione con tutor aziendale per concordare misure di adattamento.}

\risk{RP3 - Salute e malattia}
{Malattia o problematiche di salute mentale. Questa problematica potrebbe essere poco evidente dato lo svolgersi dello stage in periodo estivo ma particolarmente impattante dato il calendario fissato di giorni lavorativi.}
{1}
{Alta}
{-}
{Concordare con l'azienda giorni extra per raggiungere le ore previste.}

\risk{RO1 - Fornitura}
{Ritardi nella fornitura di risorse esterne. Mancato accesso a strumenti o risorse software necessari per lo sviluppo.}
{3}
{Alta}
{Comunicazioni rapide e chiare, misure di approvvigionamento alternative. Invio di domande e/o richieste di materiale all'azienza.}
{Comunicazione con tutor aziendale e azienza per concordare misure di adattamento.}

\risk{RO2 - Organizzazione attività inefficiente}
{Errori di valutazione in termini di tempistiche per lo svolgimento delle diverse attività pianificate.}
{2}
{Media}
{Studio a monte delle fasi di sviluppo e test dell'algoritmo.}
{Comunicazione con tutor aziendale per concordare misure di adattamento.}

\risk{RT1 - Incompatibilità tra diverse tecnologie o componenti software}
{Situazione presentante difficoltà nell’integrare le diverse tecnologie coinvolte.}
{1}
{Media}
{Studio a monte delle tecnologie e conseguente scelta ragionata delle stesse.}
{Comunicazione con tutor aziendale per concordare misure di adattamento.}

\risk{RT2 - Aggiornamenti o modifiche agli strumenti tecnologici in uso}
{Situazione nella quale alcune tecnologie individuate risultino indisponibili o aventi modifiche sostanziali potenzialmente invalidanti il lavoro svolto fino a quel punto.}
{1}
{Alta}
{Scelta ponderata delle tecnologie. Costruire un ambiente flessibile ai cambiamenti.}
{Comunicazione con tutor aziendale per concordare misure di adattamento.}
