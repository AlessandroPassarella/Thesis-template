\chapter{Processi e metodologie}
\label{cap:processi-metodologie}

\intro{In questo capitolo verranno riportati in modo approfondito lo scopo e gli obiettivi dello stage, contestualizzazione delle attività alla realtà aziendale e scadenzario delle stesse.}\\

\section{Contesto (chiara)}

\section{Introduzione al progetto}

\section{Vincoli temporali e tecnologici}
L'unico vincolo tecnologico che mi è stato imposto è stato lo sviluppo dell'algoritmo tramite il linguaggio python per favorire la semplicità dello sviluppo e rendere il processo più efficiente. Essendo un algoritmo di ottimizzazione, linguaggi come il c++ si prestano meglio, data la scarsa efficienza nell'esecuzione del codice python, tuttavia python rimane una scelta solida in vista di un possibile porting ad un linguaggio più efficiente.
Prima di iniziare lo stage è stato concordato con l’azienda un piano di lavoro su un totale di 304 ore, lavorando 5 giorni a settimana, 8 ore per ciascun giorno eccetto il venerdì che per politica aziendale prevede 6,5 ore. 

\section{Obiettivi}
Ogni obiettivo è provvisto di un codice identificativo formato da una delle seguenti stringhe \textbf{OB, DE, FA}, che rappresentano il livello di importanza e da un numero incrementale positivo, che rispetta la seguente nomenclatura:
\begin{center}
    [Importanza][Identificativo]
\end{center}
Gli obbiettivi rappresentano ciò che vogliamo ottenere dell'attività di stage e sono divisi, a seconda dell'importanza, come segue:
\begin{itemize}
    \item Obiettivi obbligatori - \textbf{OB}: 
    Rappresentano gli obiettivi vincolanti che dovranno essere necessariamente soddisfatti;
    \item Obiettivi desiderabili - \textbf{DE}:
    Rappresentano gli obiettivi non vincolanti ma dal riconoscibile valore aggiunto;
    \item Obiettivi facoltativi - \textbf{FA}:
    Rappresentano gli obiettivi facoltativi non vincolanti ma dal valore aggiunto non strettamente competitivo.
\end{itemize}
Ogni obbiettivo è affiancato da una breve descrizione.


\begin{table}[H]
\centering
\begin{tabular}{|l|p{10cm}|}
\hline
\textbf{Obiettivo} & \textbf{Descrizione} \\ \hline
OB1 & Studio e conoscenza di algoritmi euristici e metaeuristici nella letteratura della Ricerca Operativa \\ \hline
OB2 & Definizione del problema di nesting in esame \\ \hline
OB3 & Conoscenza del framework fornito dall’azienda con funzionalità di natura geometrica e di ottimizzazione nel contesto del nesting \\ \hline
OB4 & Implementazione di un algoritmo genetico per la risoluzione del problema di nesting in esame \\ \hline
OB5 & Verifica su benchmark aziendali dell’algoritmo, confrontandolo con soluzioni ottenute da altri metodi risolutivi interni all’azienda \\ \hline
OB6 & Documentazione del codice e manuale di utilizzo dei moduli sviluppati \\ \hline
\end{tabular}
\caption{Obiettivi Obbligatori}
\end{table}

\begin{table}[H]
\centering
\begin{tabular}{|l|p{10cm}|}
\hline
\textbf{Obiettivo} & \textbf{Descrizione} \\ \hline
DE1 & Analisi statistica per una validazione rigorosa e valutazione della significatività dei risultati ottenuti \\ \hline
\end{tabular}
\caption{Obiettivi Desiderabili}
\end{table}

\begin{table}[H]
\centering
\begin{tabular}{|l|p{10cm}|}
\hline
\textbf{Obiettivo} & \textbf{Descrizione} \\ \hline
FA1 & Adattamento e test dell'algoritmo genetico sviluppato su problemi di nesting simili a quello definito (ad esempio problemi con forme regolari/irregolari, strip packing) \\ \hline
\end{tabular}
\caption{Obiettivi Facoltativi}
\end{table}

\section{Pianificazione}

Divisione delle attività da svolgere con una stima in termini di ore della durata di ciascuna attività.\\

\begin{table}[H]
\centering
\begin{tabular}{|l|p{10cm}|}
\hline
\textbf{Durata in ore} & \textbf{Descrizione dell'attività} \\ \hline
30 & Studio della letteratura di Ricerca Operativa nell’ambito delle euristiche di ottimizzazione combinatoria e degli algoritmi genetici \\ \hline
10 & Studio delle tecnologie necessarie allo sviluppo dell’algoritmo di ottimizzazione (python, librerie di ottimizzazione) \\ \hline
60 & Definizione del problema di nesting e analisi degli algoritmi esistenti e della loro implementazione \\ \hline
130 & Sviluppo algoritmo genetico:
\begin{itemize}
    \item Progettazione degli operatori di codifica e decodifica delle soluzioni basati su euristiche costruttive (30 ore)
    \item Progettazione degli operatori di fitness, selezione e rimpiazzo generazionale (10 ore)
    \item Implementazione delle componenti algoritmiche (80 ore)
    \item Test degli algoritmi implementati su istanze di prova (10 ore)
\end{itemize} \\ \hline
44 & Test dell'algoritmo sviluppato su benchmark di letteratura e aziendali. Analisi statistica dei risultati prodotti su diversi benchmark confrontandoli con quelli dei metodi sviluppati dall'azienda \\ \hline
30 & Stesura documentazione codice e manuale di utilizzo dei modelli sviluppati \\ \hline
\textbf{Totale: 304 ore} & \\ \hline
\end{tabular}
\caption{Pianificazione del lavoro}
\end{table}

\section{Ambiente di lavoro}
%python, vscode, github, github desktop, jupyter notebook, librerie python
\subsection{Metodo di sviluppo}
Il ciclo di vita dell'algoritmo he seguito i seguenti passi:
\begin{itemize}
    \item Definizione del problema: (sentire nicola);
    \item Analisi letteratura: lettura articoli accademici riportanti problemi simili o idee per lo sviluppo degli stessi;
    \item Modellazione del problema: (sentire nicola);
    \item Sviluppo: codifica dell'algoritmo e integrazione con il (lavoro di chiara);
    \item Verifica: testing massivo e verifica della soluzione fornita.
\end{itemize}

\subsection{Gestione di progetto}
Per quanto riguarda la gestione di progetto sono stati utilizzati alcuni strumenti descritti con maggiore dettaglio nel Capitolo 6. In generale per la gestione della comunicazione, condivisione di informazioni, documentazione e articoli si è fatto uso dell’applicazione Microsoft Teams, per il versionamento e la gestione dei task si è fatto uso del servizio GitHub, infine per quanto riguarda l’interfaccia con cui versionare ho utilizzato GitHub Desktop.

\subsection{Linguaggio di programmazione e ambiente di sviluppo}
Per la fase iniziale della codifica si è lavorato utilizzando Jupyter notebook oppure, con il quale è stato possibile scrivere metodi in linguaggio Python in modo molto agevole. Quando l'algoritmo ha raggiunto una certa completezza è stato necessario raggruppare i vari metodi in più classi e iniziare a lavorare su VS Code all'algoritmo e su Jupyter Notebook ai test da effettuare.
Python è un linguaggio di programmazione orientato agli oggetti ed interpretato dinamicamente al momento dell’esecuzione da un interprete. Python risulta veramente versatile in quanto fornisce incredibili funzionalità utilizzabili in modo semplice e intuitivo, inoltre, dispone di moltissime librerie che permettono le più svariate operazioni, come: metodi per l'analisi statistica, la manipolazione dei dati, la creazione di grafici e operazioni sui file.

\section{Analisi dei rischi}

\begin{risk}{Performance del simulatore hardware}
    \riskdescription{le performance del simulatore hardware e la comunicazione con questo potrebbero risultare lenti o non abbastanza buoni da causare il fallimento dei test}
    \risksolution{coinvolgimento del responsabile a capo del progetto relativo il simulatore hardware}
    \label{risk:hardware-simulator} 
\end{risk}