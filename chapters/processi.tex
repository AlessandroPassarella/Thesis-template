\chapter{Processi e metodologie}
\label{cap:processi-metodologie}

\intro{In questo capitolo verranno riportati in modo approfondito lo scopo e gli obiettivi dello stage, contestualizzazione delle attività alla realtà aziendale e scadenzario delle stesse.}\\

\section{Contesto}

Nel campo della Ricerca Operativa, il Cutting Stock Problem (CSP) rappresenta una classe di problemi di ottimizzazione combinatoria di rilevanza pratica che coinvolge l'assegnazione di oggetti più piccoli a oggetti più grandi, generalmente con l'obiettivo di minimizzare gli scarti. I problemi di CSP possono differire tra loro per una grande varietà di caratteristiche. Ad esempio, le dimensioni sia degli oggetti piccoli che di quelli grandi possono variare da 1 a 3 dimensioni. Il caso bidimensionale riguarda il taglio di oggetti (o parti) da fogli di materiale riducendo gli scarti, come avviene nelle industrie del legno, del vetro e della lamiera.

\subsection{Definizione del problema}

Di seguito, ci concentreremo sulla classe bidimensionale dei problemi di CSP, anche nota come Nesting. Nella maggior parte dei problemi di Nesting, una diretta conseguenza del fatto che gli oggetti vengono tagliati dai fogli di materiale è che, nel costruire una disposizione di taglio posizionando gli oggetti all'interno del foglio, essi non possono sovrapporsi tra loro e devono essere completamente contenuti nel foglio stesso. Pertanto, il compito di ottimizzazione di un problema di Nesting comporta generalmente  evitare la sovrapposizione tra le forme nel piano rispettando determinati vincoli. I vincoli geometrici e la natura combinatoria di questi tipi di problemi li rendono difficili da risolvere in modo ottimale nella maggior parte dei casi. Inoltre, a seconda dell'applicazione, la dimensione delle istanze può variare ampiamente e diventare molto grande così come le dimensioni delle parti che compongono le istanze, motivo per cui sono stati sviluppati diversi approcci metaeuristici per affrontare questa classe di problemi.

Nel nostro caso occorre tenere conto di diversi vincoli, alcuni classici del Nesting altri specifici per il caso Salvagnini:
\begin{itemize}
    \item Non sovrapposizione: gli elementi posizionati su uno stesso foglio non devono sovrapporsi;
    \item Margini di punzonatura: i margini di punzonatura assegnati devono essere rispettati su tutti i lati di ogni elemento;
    \item Margini di sicurezza o taglio comune: se gli elementi adiacenti non condividono un bordo (taglio comune, sovrapposizione del margine), devono rispettare il margine di sicurezza;
    \item Vincoli sulle assegnazioni: gli elementi obbligatori devono necessariamente essere posizionati tutti, mentre quelli opzionali sono utilizzati, anche solo in parte, solo per ridurre gli scarti;
    \item Vincoli di hard precedence: gli elementi con precedenza hard più alta devono essere posizionati prima di quelli con precedenza più bassa, se presenti (precedenza che non può essere infranta);
    \item Vinvoli di soft precedence: se due elementi hanno una relazione di precedenza soft, questa va rispettata solo se non comporta un aumento degli scarti (precedenza che può essere infranta);
    \item Rotazioni consentite: ciascun elemento può essere ruotato solo secondo le angolazioni permesse dagli strumenti di punzonatura.
\end{itemize}


\section{Vincoli temporali e tecnologici}
L'unico vincolo tecnologico che mi è stato imposto è stato lo sviluppo dell'algoritmo tramite il linguaggio Python. Questo perché python rendere il processo di sviluppo più efficiente, infatti, è particolarmente adatto al fast prototyping, dato che non necessita di compilazione, a differenza di linguaggi come C++ e Java, e, inoltre, permette di avere maggiore flessibilità del codice grazie anche dall'uso dei notebook.

Prima di iniziare lo stage è stato concordato con l’azienda un piano di lavoro su un totale di 304 ore, lavorando 5 giorni a settimana, 8 ore per ciascun giorno eccetto il venerdì che per politica aziendale prevede 6,5 ore. 

\section{Obiettivi}
Ogni obiettivo è provvisto di un codice identificativo formato da una delle seguenti stringhe \textbf{OB}, \textbf{DE} e \textbf{FA}, che rappresentano il livello di importanza e da un numero incrementale positivo, che rispetta la seguente nomenclatura:
\begin{center}
    [Importanza][Identificativo]
\end{center}
Gli obbiettivi rappresentano ciò che vogliamo ottenere dell'attività di stage e sono divisi, a seconda dell'importanza, come segue:
\begin{itemize}
    \item Obiettivi obbligatori \textbf{(OB)}: 
    Rappresentano gli obiettivi vincolanti che dovranno essere necessariamente soddisfatti;
    \item Obiettivi desiderabili \textbf{(DE)}:
    Rappresentano gli obiettivi non vincolanti ma dal riconoscibile valore aggiunto;
    \item Obiettivi facoltativi \textbf{(FA)}:
    Rappresentano gli obiettivi facoltativi non vincolanti ma dal valore aggiunto non strettamente competitivo.
\end{itemize}
Ogni obbiettivo è affiancato da una breve descrizione.


\begin{table}[H]
\centering
\begin{tabular}{|l|p{10cm}|}
\hline
\textbf{Obiettivo} & \textbf{Descrizione} \\ \hline
OB1 & Studio e conoscenza di algoritmi euristici e metaeuristici nella letteratura della Ricerca Operativa \\ \hline
OB2 & Definizione del problema di nesting in esame \\ \hline
OB3 & Conoscenza del framework fornito dall’azienda con funzionalità di natura geometrica e di ottimizzazione nel contesto del nesting \\ \hline
OB4 & Implementazione di un algoritmo genetico per la risoluzione del problema di nesting in esame \\ \hline
OB5 & Verifica su benchmark aziendali dell’algoritmo, confrontandolo con soluzioni ottenute da altri metodi risolutivi interni all’azienda \\ \hline
OB6 & Documentazione del codice e manuale di utilizzo dei moduli sviluppati \\ \hline
\end{tabular}
\caption{Obiettivi Obbligatori}
\end{table}

\begin{table}[H]
\centering
\begin{tabular}{|l|p{10cm}|}
\hline
\textbf{Obiettivo} & \textbf{Descrizione} \\ \hline
DE1 & Analisi statistica per una validazione rigorosa e valutazione della significatività dei risultati ottenuti \\ \hline
\end{tabular}
\caption{Obiettivi Desiderabili}
\end{table}

\begin{table}[H]
\centering
\begin{tabular}{|l|p{10cm}|}
\hline
\textbf{Obiettivo} & \textbf{Descrizione} \\ \hline
FA1 & Adattamento e test dell'algoritmo genetico sviluppato su problemi di nesting simili a quello definito (ad esempio problemi con forme regolari/irregolari, strip packing) \\ \hline
\end{tabular}
\caption{Obiettivi Facoltativi}
\end{table}

\section{Pianificazione}

Divisione delle attività da svolgere con una stima in termini di ore della durata di ciascuna attività.\\

\begin{table}[H]
\centering
\begin{tabular}{|l|p{10cm}|}
\hline
\textbf{Durata in ore} & \textbf{Descrizione dell'attività} \\ \hline
30 & Studio della letteratura di Ricerca Operativa nell’ambito delle euristiche di ottimizzazione combinatoria e degli algoritmi genetici \\ \hline
10 & Studio delle tecnologie necessarie allo sviluppo dell’algoritmo di ottimizzazione (python, librerie di ottimizzazione) \\ \hline
60 & Definizione del problema di nesting e analisi degli algoritmi esistenti e della loro implementazione \\ \hline
130 & Sviluppo algoritmo genetico:
\begin{itemize}
    \item Progettazione degli operatori di codifica e decodifica delle soluzioni basati su euristiche costruttive (30 ore)
    \item Progettazione degli operatori di fitness, selezione e rimpiazzo generazionale (10 ore)
    \item Implementazione delle componenti algoritmiche (80 ore)
    \item Test degli algoritmi implementati su istanze di prova (10 ore)
\end{itemize} \\ \hline
44 & Test dell'algoritmo sviluppato su benchmark di letteratura e aziendali. Analisi statistica dei risultati prodotti su diversi benchmark confrontandoli con quelli dei metodi sviluppati dall'azienda \\ \hline
30 & Stesura documentazione codice e manuale di utilizzo dei modelli sviluppati \\ \hline
\textbf{Totale: 304 ore} & \\ \hline
\end{tabular}
\caption{Pianificazione del lavoro}
\end{table}

\section{Ambiente di lavoro}
Di seguito verranno esposti brevemente il ciclo di vita e la gestione del progetto, considerando anche l'ambiente di sviluppo.
%python, vscode, github, github desktop, jupyter notebook, librerie python
\subsection{Metodo di sviluppo}
Il ciclo di vita dell'algoritmo he seguito i seguenti passi:
\begin{itemize}
    \item Comprensione del problema: definizione del problema di nesting;
    \item Analisi letteratura: lettura articoli accademici di algoritmi genetici;
    \item Sviluppo: codifica dell'algoritmo e integrazione con il (lavoro di chiara);
    \item Verifica: testing massivo e verifica della soluzione fornita.
\end{itemize}

\subsection{Gestione di progetto}
Per quanto riguarda la gestione di progetto sono stati utilizzati alcuni strumenti descritti con maggiore dettaglio nel Capitolo 5. In generale per la gestione della comunicazione, condivisione di informazioni, documentazione e articoli si è fatto uso dell’applicazione Microsoft Teams, per il versionamento e la gestione dei task si è fatto uso del servizio GitHub, infine per quanto riguarda l’interfaccia con cui versionare ho utilizzato GitHub Desktop.

\subsection{Linguaggio di programmazione e ambiente di sviluppo}
Per la fase iniziale della codifica si è lavorato utilizzando Jupyter notebook oppure, con il quale è stato possibile scrivere metodi in linguaggio Python in modo molto agevole. Quando l'algoritmo ha raggiunto una certa completezza è stato necessario raggruppare i vari metodi in più classi e iniziare a lavorare su VS Code all'algoritmo e su Jupyter Notebook ai test da effettuare.
Python è un linguaggio di programmazione orientato agli oggetti ed interpretato dinamicamente al momento dell’esecuzione da un interprete. Python risulta veramente versatile in quanto fornisce incredibili funzionalità utilizzabili in modo semplice e intuitivo, inoltre, dispone di moltissime librerie che permettono le più svariate operazioni, come: metodi per l'analisi statistica, la manipolazione dei dati, la creazione di grafici e operazioni sui file.

\section{Analisi dei rischi}

Questa sezione del documento si concentra sull’analisi delle potenziali difficoltà che potrebbero emergere durante il corso dello stage, con l’obiettivo di identificare, analizzare e prevenire eventuali ostacoli o rallentamenti che potrebbero comprometterne il progresso complessivo. Per affrontare queste possibili problematiche, si è scelto di esaminare attentamente ciascun rischio, fornendo dettagli quali la descrizione del rischio, il grado di rischio associato, la pericolosità, le precauzioni da adottare e una possibile soluzione. Il grado di rischio definisce la possibilità di occorrenza dello stesso e varia da una scala crescente da 1 a 5, mentre la pericolosità varia tra “Alta”, “Media” e “Bassa”. 

Le informazioni vengono presentante in forma tabellare, in modo da facilitarne il monitoraggio continuo durante l’intero ciclo di vita del progetto. Ogni rischio è provvisto di un codice identificativo formato da una delle seguenti stringhe \textbf{RP}, \textbf{RO} e \textbf{RT}, che rappresentano la categoria di rischo e da un numero incrementale positivo. Le principali categorie di rischi considerate includono:
\begin{itemize}
    \item Rischi personali \textbf{(RP)};
    \item Rischi organizzativi \textbf{(RO)};
    \item Rischi tecnologici/software \textbf{(RT)}.
\end{itemize}
I rischi rispettano la seguente nomenclatura:
\begin{center}
    [Categoria][Identificativo]
\end{center}

\risk{RP1 - Mancanza di competenze tecniche specifiche}
{Mancanza di esperienze professionali nella realizzazione dell'algoritmo. Mancanza di conoscenze delle tecnologie necessarie allo sviluppo del prodotto.}
{2}
{Alta}
{Fase iniziale di studio delle tecnologie.}
{Comunicazione con tutor aziendale per concordare misure di adattamento.}

\risk{RP2 - Mancanza di conoscenze teoriche specifiche}
{Mancanza di conoscenze relative al problema in questione. Mancanza di conoscenza specifiche dell'ambito industriale.}
{5}
{Media}
{Fase iniziale di studio del problema di nesting e dell'approccio aziendale.}
{Comunicazione con tutor aziendale per concordare misure di adattamento.}

\risk{RP3 - Salute e malattia}
{Malattia o problematiche di salute mentale. Questa problematica potrebbe essere poco evidente dato lo svolgersi dello stage in periodo estivo ma particolarmente impattante dato il calendario fissato di giorni lavorativi.}
{1}
{Alta}
{-}
{Concordare con l'azienda giorni extra per raggiungere le ore previste.}

\risk{RO1 - Fornitura}
{Ritardi nella fornitura di risorse esterne. Mancato accesso a strumenti o risorse software necessari per lo sviluppo.}
{3}
{Alta}
{Comunicazioni rapide e chiare, misure di approvvigionamento alternative. Invio di domande e/o richieste di materiale all'azienza.}
{Comunicazione con tutor aziendale e azienza per concordare misure di adattamento.}

\risk{RO2 - Organizzazione attività inefficiente}
{Errori di valutazione in termini di tempistiche per lo svolgimento delle diverse attività pianificate.}
{2}
{Media}
{Studio a monte delle fasi di sviluppo e test dell'algoritmo.}
{Comunicazione con tutor aziendale per concordare misure di adattamento.}

\risk{RT1 - Incompatibilità tra diverse tecnologie o componenti software}
{Situazione presentante difficoltà nell’integrare le diverse tecnologie coinvolte.}
{1}
{Media}
{Studio a monte delle tecnologie e conseguente scelta ragionata delle stesse.}
{Comunicazione con tutor aziendale per concordare misure di adattamento.}

\risk{RT2 - Aggiornamenti o modifiche agli strumenti tecnologici in uso}
{Situazione nella quale alcune tecnologie individuate risultino indisponibili o aventi modifiche sostanziali potenzialmente invalidanti il lavoro svolto fino a quel punto.}
{1}
{Alta}
{Scelta ponderata delle tecnologie. Costruire un ambiente flessibile ai cambiamenti.}
{Comunicazione con tutor aziendale per concordare misure di adattamento.}
