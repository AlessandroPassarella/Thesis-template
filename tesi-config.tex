%**************************************************************
% file contenente le impostazioni della tesi
%**************************************************************

%**************************************************************
% Frontespizio
%**************************************************************

% Autore
\newcommand{\myName}{Riccardo Montagnin}                                    
\newcommand{\myTitle}{Titolo della tesi}

% Tipo di tesi                   
\newcommand{\myDegree}{Tesi di laurea}

% Università             
\newcommand{\myUni}{Università degli Studi di Padova}

% Facoltà       
\newcommand{\myFaculty}{Corso di Laurea in Informatica}

% Dipartimento
\newcommand{\myDepartment}{Dipartimento di Matematica "Tullio Levi-Civita"}

% Titolo del relatore
\newcommand{\profTitle}{Prof.}

% Relatore
\newcommand{\myProf}{Tullio Vardanega}

% Luogo
\newcommand{\myLocation}{Padova}

% Anno accademico
\newcommand{\myAA}{2016-2017}

% Data discussione
\newcommand{\myTime}{Luglio 2017}


%**************************************************************
% Impostazioni di impaginazione
% see: http://wwwcdf.pd.infn.it/AppuntiLinux/a2547.htm
%**************************************************************

\setlength{\parindent}{14pt}   % larghezza rientro della prima riga
\setlength{\parskip}{0pt}   % distanza tra i paragrafi


%**************************************************************
% Impostazioni di biblatex
%**************************************************************
\bibliography{bibliografia} % database di biblatex 

\defbibheading{bibliography} {
    \cleardoublepage
    \phantomsection 
    \addcontentsline{toc}{chapter}{\bibname}
    \chapter*{\bibname\markboth{\bibname}{\bibname}}
}

\setlength\bibitemsep{1.5\itemsep} % spazio tra entry

\DeclareBibliographyCategory{opere}
\DeclareBibliographyCategory{web}

\addtocategory{opere}{womak:lean-thinking}
\addtocategory{web}{site:agile-manifesto}

\defbibheading{opere}{\section*{Riferimenti bibliografici}}
\defbibheading{web}{\section*{Siti Web consultati}}


%**************************************************************
% Impostazioni di caption
%**************************************************************
\captionsetup{
    tableposition=top,
    figureposition=bottom,
    font=small,
    format=hang,
    labelfont=bf
}

%**************************************************************
% Impostazioni di glossaries
%**************************************************************
\makeglossaries
\cleardoublepage
\chapter{Glossario}

\begin{description}
    \item[Benchmark] \hfill \\
    Riferimento utilizzato per valutare le prestazioni di un algoritmo confrontandolo con soluzioni standard o con risultati ottenuti in casi di studio precedenti.

    \item[Black-box] \hfill \\
    In ambito informatico, un sistema black box, similmente ad una scatola nera, è descrivibile essenzialmente nel suo comportamento esterno ovvero solo per come reagisce in uscita (output) a una determinata sollecitazione in ingresso (input), ma il cui funzionamento interno è non visibile o ignoto.

    \item[Branch-and-Bound] \hfill \\
    Algoritmo per risolvere problemi di ottimizzazione combinatoria dividendo iterativamente il problema in sottoproblemi più piccoli, escludendo quelli che non possono contenere soluzioni ottimali.

    \item[CSV] \hfill \\
    Formato di file (Comma-Separated Values) utilizzato per rappresentare dati strutturati in forma tabellare, dove ogni riga è un record e ogni campo è separato da una virgola.

    \item[Euristiche] \hfill \\
    Procedure semplificate e specifiche per la risoluzione di problemi complessi, che producono soluzioni accettabili in tempi brevi ma non garantiscono l'ottimalità.

    \item[Fast Prototyping] \hfill \\
    Metodo di sviluppo software rapido che consente di creare prototipi funzionali in tempi brevi, facilitando iterazioni e miglioramenti successivi.

    \item[Framework] \hfill \\
    Struttura software che fornisce un insieme predefinito di funzionalità e strumenti per facilitare lo sviluppo di applicazioni, ad esempio gestendo attività ripetitive o comuni.

    \item[GitHub] \hfill \\
    Piattaforma di controllo di versione e collaborazione basata su Git, utilizzata per gestire codice sorgente, tracciare modifiche e coordinare il lavoro su progetti software.

    \item[GitHub Desktop] \hfil \\
    Applicazione desktop per la gestione dei repository Git su GitHub, che semplifica operazioni come il commit, il pull e il push, fornendo un'interfaccia grafica intuitiva per il controllo di versione.

    \item[Jupyter Notebook] \hfil \\
    Ambiente interattivo basato su browser che consente di combinare codice eseguibile, testo esplicativo, visualizzazioni e output in un unico documento, particolarmente utile per analisi di dati e prototipazione rapida.

    \item[Margini di sicurezza] \hfill \\ \hypertarget{gls2}{}
    Spazio minimo richiesto tra elementi o dai bordi di un foglio, per garantire la corretta esecuzione delle operazioni di lavorazione, come il taglio o la punzonatura, senza compromettere la qualità.

    \item[Margini per il taglio comune] \hfill \\
    Spazio dedicato tra due elementi adiacenti sullo stesso foglio, utilizzato per garantire che il taglio li separi correttamente senza sovrapposizioni o interferenze.

    \item[Metaeuristiche] \hfill \\
    Strategie generali di ottimizzazione che combinano esplorazione e intensificazione per risolvere problemi complessi, spesso NP-hard, in modo approssimato e in tempi ragionevoli.

    \item[Microsoft Teams] \hfill \\
    Piattaforma collaborativa di Microsoft che integra strumenti per la comunicazione (chat e videochiamate), la condivisione di file e la gestione dei progetti.

    \item[Notebook] \hfill \\
    Ambiente interattivo per la programmazione, come Jupyter Notebook, che consente di combinare codice eseguibile, testo esplicativo, visualizzazioni e output in un unico documento.

    \item[NP-hard] \hfill \\
    Classe di problemi decisionali in teoria della complessità che richiedono tempo esponenziale per essere risolti nel caso peggiore, a meno che $P=NP$. Include problemi che non possono essere risolti in modo efficiente.

    \item[Overfitting] \hfill \\
    Fenomeno in cui un algoritmo è ottimizzato eccessivamente per specifiche istanze o dati, perdendo generalizzazione e diventando meno efficace su nuovi problemi.

    \item[Problemi di ottimizzazione combinatoria] \hfill \\
    Problemi matematici che consistono nel trovare una soluzione ottimale da un insieme discreto o finito di possibilità, spesso soggetti a vincoli. Esempi comuni includono il problema del commesso viaggiatore e il problema dello zaino.

    \item[Programmazione lineare intera mista] \hfill \\
    Tecnica di ottimizzazione matematica che consente di risolvere problemi in cui alcune variabili devono assumere valori interi, mentre altre possono essere continue, e l'obiettivo e i vincoli sono espressi tramite funzioni lineari.

    \item[Python] \hfill \\
    Linguaggio di programmazione ad alto livello, orientato agli oggetti e interpretato, noto per la sua sintassi leggibile e per la disponibilità di librerie potenti, che lo rendono ideale per prototipazione rapida e analisi dei dati.

    \item[Scarto] \hfill \\
    Materiale rimanente inutilizzato dopo il processo di taglio o lavorazione, che rappresenta una perdita da minimizzare nei problemi di ottimizzazione del nesting.

    \item[T-test] \hfill \\ \hypertarget{gls4}{}
    Test statistico utilizzato per confrontare le medie di due gruppi e determinare se le differenze osservate sono statisticamente significative [\hyperlink{bibliografia}{8}].

    \item[VisualStudio Code] \hfill \\
    Editor di codice sorgente multipiattaforma sviluppato da Microsoft, noto per la sua velocità, flessibilità e l'ampia disponibilità di estensioni per il supporto di vari linguaggi di programmazione.
\end{description}

 % database di termini


%**************************************************************
% Impostazioni di graphicx
%**************************************************************
\graphicspath{{immagini/}} % cartella dove sono riposte le immagini


%**************************************************************
% Impostazioni di hyperref
%**************************************************************
\hypersetup{
    %hyperfootnotes=false,
    %pdfpagelabels,
    %draft,	% = elimina tutti i link (utile per stampe in bianco e nero)
    colorlinks=true,
    linktocpage=true,
    pdfstartpage=1,
    pdfstartview=,
    % decommenta la riga seguente per avere link in nero (per esempio per la stampa in bianco e nero)
    %colorlinks=false, linktocpage=false, pdfborder={0 0 0}, pdfstartpage=1, pdfstartview=FitV,
    breaklinks=true,
    pdfpagemode=UseNone,
    pageanchor=true,
    pdfpagemode=UseOutlines,
    plainpages=false,
    bookmarksnumbered,
    bookmarksopen=true,
    bookmarksopenlevel=1,
    hypertexnames=true,
    pdfhighlight=/O,
    %nesting=true,
    %frenchlinks,
    urlcolor=webbrown,
    linkcolor=RoyalBlue,
    citecolor=webgreen,
    %pagecolor=RoyalBlue,
    %urlcolor=Black, linkcolor=Black, citecolor=Black, %pagecolor=Black,
    pdftitle={\myTitle},
    pdfauthor={\textcopyright\ \myName, \myUni, \myFaculty},
    pdfsubject={},
    pdfkeywords={},
    pdfcreator={pdfLaTeX},
    pdfproducer={LaTeX}
}

%**************************************************************
% Impostazioni di itemize
%**************************************************************
\renewcommand{\labelitemi}{$\ast$}

%\renewcommand{\labelitemi}{$\bullet$}
%\renewcommand{\labelitemii}{$\cdot$}
%\renewcommand{\labelitemiii}{$\diamond$}
%\renewcommand{\labelitemiv}{$\ast$}


%**************************************************************
% Impostazioni di listings
%**************************************************************
\lstset{
    language=[LaTeX]Tex,%C++,
    keywordstyle=\color{RoyalBlue}, %\bfseries,
    basicstyle=\small\ttfamily,
    %identifierstyle=\color{NavyBlue},
    commentstyle=\color{Green}\ttfamily,
    stringstyle=\rmfamily,
    numbers=none, %left,%
    numberstyle=\scriptsize, %\tiny
    stepnumber=5,
    numbersep=8pt,
    showstringspaces=false,
    breaklines=true,
    frameround=ftff,
    frame=single
} 


%**************************************************************
% Impostazioni di xcolor
%**************************************************************
\definecolor{webgreen}{rgb}{0,.5,0}
\definecolor{webbrown}{rgb}{.6,0,0}


%**************************************************************
% Altro
%**************************************************************

\newcommand{\omissis}{[\dots\negthinspace]} % produce [...]

% eccezioni all'algoritmo di sillabazione
\hyphenation
{
    ma-cro-istru-zio-ne
    gi-ral-din
}

\newcommand{\sectionname}{sezione}
\addto\captionsitalian{\renewcommand{\figurename}{Figura}
                       \renewcommand{\tablename}{Tabella}}

\newcommand{\glsfirstoccur}{\ap{{[g]}}}

\newcommand{\intro}[1]{\emph{\textsf{#1}}}

%**************************************************************
% Environment per ``rischi''
%**************************************************************
\newcounter{riskcounter}                % define a counter
\setcounter{riskcounter}{0}             % set the counter to some initial value

%%%% Parameters
% #1: Title
\newenvironment{risk}[1]{
    \refstepcounter{riskcounter}        % increment counter
    \par \noindent                      % start new paragraph
    \textbf{\arabic{riskcounter}. #1}   % display the title before the 
                                        % content of the environment is displayed 
}{
    \par\medskip
}

\newcommand{\riskname}{Rischio}

\newcommand{\riskdescription}[1]{\textbf{\\Descrizione:} #1.}

\newcommand{\risksolution}[1]{\textbf{\\Soluzione:} #1.}

%**************************************************************
% Environment per ``use case''
%**************************************************************
\newcounter{usecasecounter}             % define a counter
\setcounter{usecasecounter}{0}          % set the counter to some initial value

%%%% Parameters
% #1: ID
% #2: Nome
\newenvironment{usecase}[2]{
    \renewcommand{\theusecasecounter}{\usecasename #1}  % this is where the display of 
                                                        % the counter is overwritten/modified
    \refstepcounter{usecasecounter}             % increment counter
    \vspace{10pt}
    \par \noindent                              % start new paragraph
    {\large \textbf{\usecasename #1: #2}}       % display the title before the 
                                                % content of the environment is displayed 
    \medskip
}{
    \medskip
}

\newcommand{\usecasename}{UC}

\newcommand{\usecaseactors}[1]{\textbf{\\Attori Principali:} #1. \vspace{4pt}}
\newcommand{\usecasepre}[1]{\textbf{\\Precondizioni:} #1. \vspace{4pt}}
\newcommand{\usecasedesc}[1]{\textbf{\\Descrizione:} #1. \vspace{4pt}}
\newcommand{\usecasepost}[1]{\textbf{\\Postcondizioni:} #1. \vspace{4pt}}
\newcommand{\usecasealt}[1]{\textbf{\\Scenario Alternativo:} #1. \vspace{4pt}}

%**************************************************************
% Environment per ``namespace description''
%**************************************************************

\newenvironment{namespacedesc}{
    \vspace{10pt}
    \par \noindent                              % start new paragraph
    \begin{description} 
}{
    \end{description}
    \medskip
}

\newcommand{\classdesc}[2]{\item[\textbf{#1:}] #2}
