\newcommand{\myName}{Alessandro Passarella}
\newcommand{\myID}{2000557}
\newcommand{\myTitle}{Progetto e Implementazione di un Algoritmo Genetico per l'Ottimizzazione del Nesting di Forme Regolari su Lamiere Metalliche}
\newcommand{\myDegree}{Tesi di laurea}
\newcommand{\myUni}{Università degli Studi di Padova}
% For BSc level just use "Corso di Laurea" and don't add "Triennale" to it
\newcommand{\myFaculty}{Corso di Laurea in Informatica}
\newcommand{\myDepartment}{Dipartimento di Matematica ``Tullio Levi-Civita''}
\newcommand{\profTitle}{Prof.}
\newcommand{\myProf}{Luigi De Giovanni}
\newcommand{\myLocation}{Padova}
\newcommand{\myAA}{2023-2024}
\newcommand{\myTime}{Dicembre 2024}

% PDF file metadata fields
% when updating them delete the build directory, otherwise they won't change
\begin{filecontents*}{\jobname.xmpdata}
  \Title{Progetto e Implementazione di un Algoritmo Genetico per l'Ottimizzazione del Nesting di Forme Regolari su Lamiere Metalliche}
  \Author{Alessandro Passarella}
  \Language{it-IT}
  \Subject{Progettare e implementare un algoritmo genetico per ottimizzare il processo di nesting di forme regolari su lamiere metalliche. Il nesting è una tecnica utilizzata per disporre in modo efficiente elementi geometrici su superfici limitate, riducendo al minimo gli scarti di materiale.}
  \Keywords{Algoritmo Genetico\sep Ricerca Operativa\sep Nesting}
\end{filecontents*}
