\cleardoublepage
\phantomsection
\pdfbookmark{Sommario}{Sommario}
\begingroup
\let\clearpage\relax
\let\cleardoublepage\relax
\let\cleardoublepage\relax

\chapter*{Sommario}

Il presente documento si propone di descrivere il lavoro svolto durante il periodo di stage, della durata di circa trecento ore, dal laureando Alessandro Passarella presso l'azienda Salvagnini S.p.A. Lo scopo dello stage era quello di progettare e implementare un algoritmo genetico per ottimizzare il processo di nesting di forme regolari su lamiere metalliche. Il nesting è una tecnica utilizzata per disporre in modo efficiente elementi geometrici su superfici limitate, riducendo al minimo gli scarti di materiale. 

Il lavoro si articola in diverse fasi: inizialmente viene condotto uno studio della letteratura sugli algoritmi genetici, spesso utilizzati nella risoluzione di problemi di ottimizzazione combinatoria. Successivamente, si analizza il problema specifico di nesting affrontato dall'azienda e il codice esistente utilizzato per la sua risoluzione. Si procede quindi alla progettazione e implementazione di un algoritmo genetico che si integri con il software di ottimizzazione aziendale. Infine, i risultati ottenuti dall'algoritmo vengono confrontati con quelli di altre tecniche utilizzate dall'azienda attraverso un'analisi statistica, valutandone l'efficacia in termini di qualità delle soluzioni e tempi di esecuzione.

Questo lavoro offre un contributo all'ottimizzazione dei processi di produzione nel settore della lavorazione delle lamiere, migliorando l'efficienza e riducendo gli sprechi di materiale.

%\vfill

%\selectlanguage{english}
%\pdfbookmark{Abstract}{Abstract}
%\chapter*{Abstract}

%\selectlanguage{italian}

\endgroup

\vfill
